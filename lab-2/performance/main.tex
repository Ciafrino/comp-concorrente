\documentclass[11pt,oneside,openright,numbers=noenddot]{scrbook}
\usepackage[T1]{fontenc}
\usepackage[utf8x]{inputenc}
\usepackage[portuguese]{babel}
\usepackage{amsmath}
\usepackage{amssymb}
\usepackage{tikz}
\usepackage{tikz-3dplot}
\usepackage{color}
\usepackage{transparent}
\usepackage{pgfplots}
\usepackage{hyperref}
\usepackage{tcolorbox}
\usepackage{xspace}

\author{Christopher Ciafrino de Souza}
\title{Avaliação de aplicações concorrentes - Laboratório 2}
\date{\today}

\begin{document}
\maketitle

\section{Testes}
Executei cinco vezes cada um dos testes abaixo 
numa matriz $N$ por $N$ preenchida com 1's. 

\begin{enumerate}
	\item Com $N = 500$
	\begin{enumerate}
		\item Uma thread:\\
			Tempo inicializaçao: 0.001649\\
			Tempo execução: 0.633234\\
			Tempo finalização: 0.000166
		\item Duas threads:\\
			Tempo inicializaçao: 0.001665\\
			Tempo execução: 0.372172\\
			Tempo finalização: 0.000204		
		\item Quatro threads:\\
			Tempo inicializaçao: 0.001515\\
			Tempo execução: 0.359666\\
			Tempo finalização: 0.000191
	\end{enumerate}
	\item Com $N = 1000$
	\begin{enumerate}
		\item Uma thread:\\
			Tempo inicializaçao: 0.006679\\
			Tempo execução: 5.824089\\
			Tempo finalização: 0.000674	
		\item Duas threads:\\
			Tempo inicializaçao: 0.006140\\
			Tempo execução: 3.499523\\
			Tempo finalização: 0.000815
		\item Quatro threads:\\
			Tempo inicializaçao: 0.012762\\
			Tempo execução: 3.998720\\
			Tempo finalização: 0.000698
	\end{enumerate}

\item Com $N = 2000$
	\begin{enumerate}
		\item Uma thread:\\
			Tempo inicializaçao: 0.024879\\
			Tempo execução: 72.888006\\
			Tempo finalização: 0.003429
		\item Duas threads:\\
			Tempo inicializaçao: 0.023512\\
			Tempo execução: 59.501104\\
			Tempo finalização: 0.004463
		\item Quatro threads:\\
			Tempo inicializaçao: 0.024411\\
			Tempo execução: 53.986719\\
			Tempo finalização: 0.004337
	\end{enumerate}
	
\end{enumerate}

\section{Conclusão}

O valores batem sim com o que eu esperava. Conforme fui aumentando a dimensão
da matriz consegui ver como o aumento na quantidade de threads
impactava mais no desempenho do programa (como pode ser observado nos resultados acima).

\


Executei esses testes na minha máquina pessoal com processador core i7-4510U 
que possui 4 CPU's e 2 threads por core.

\end{document}
